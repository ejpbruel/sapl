\section{The \Sapl Programming Language and Interpreter}
\label{sapljs:sec:sapl}
\Sapl stands for \textbf{S}imple \textbf{A}pplication \textbf{P}rogramming
\textbf{L}anguage. The basic version of \Sapl provides no special constructs for
algebraic data types. Instead, they are encoded as ordinary functions.  Details 
on this encoding and its resulting complexity can be found \cite{JKP}.
Here we only give a few examples to show how it is realized.

We start with the encoding of the list data type, together with the \texttt{sum}function:
\begin{CleanCode}
Nil       f1  f2 = f1
Cons x xs f1  f2 = f2 x xs
sum          xxs = select xxs 0 (\x xs = x + sum xs)
\end{CleanCode}

The \texttt{select} keyword acts as a hint to the compiler that \texttt{xxs} is 
a constructor, which allows the generation of more efficient code (again see
\cite{JKP} for the details). Semantically, \texttt{select} acts as the identity
function. The remaining arguments are functions that act on the arguments of a
constructor (analog to clauses in a case-statement).

As a more complex example, consider a \Haskell function such as
\texttt{mappair}, which is based on the use of pattern matching:
  
\begin{CleanCode}
mappair f Nil          yys          = Nil 
mappair f (Cons x xs)  Nil          = Nil 
mappair f (Cons x xs)  (Cons y ys)  = Cons (f x y) (mappair f xs ys) 
\end{CleanCode}
This definition is transformed to the following \Sapl expression (using the
above definitions for \emph{Nil} and \emph{Cons}).
\begin{CleanCode}
mappair f xxs yys = select xxs Nil
                               (\x xs = select xs Nil
                                               (\y ys = Cons (f x y)
                                                             (mappair f xs ys)))
\end{CleanCode}
%
\Sapl is used as an intermediate formalism for the interpretation of non-strict
purely functional programming languages such as \Haskell and \Clean. For \Clean,
we added a \Sapl back-end to the \Clean compiler that generates \Sapl code which
the interpreter can run. Recently, the \Clean compiler has been extended to
handle the compilation of \Haskell programs as well \cite{HASCLEAN}.

The \Sapl interpreter has been implemented uon both \textsf{C} and \Java. The
\Java can be loaded as an into browsers that have the \Java Virtual Machine
(JVM) installed, and can be used for execution of \iTask tasks on the client
side.

\subsection{Some remarks on the definition of \Sapl}
\Sapl is very similar to the core languages of \Haskell and \Clean. 
Therefore, we chose not give a full definition of its syntax and semantics.
Rather, we only say something about its main characteristics and give a few
examples to illustrate these.

The only keywords in \Sapl are \texttt{let}, \texttt{if} and \texttt{select}.
Only constant \texttt{let} expressions are allowed (that may be cyclic, \Sapl
has no separate \texttt{letrec}). These may occur at the top level in a functionand at the top level in arguments of an \texttt{if} and \texttt{select}.
\texttt{where} clauses are not allowed. $\lambda$-expressions may only occur as 
arguments to a \texttt{select}. All other $\lambda$-expressions should be
lifted to the top level. 

For readability we adopted a \Clean like type definition style in \Sapl. This
also allows for the generation of more efficient code (as will become apparent
in section \ref{sapl:sec:sapljs}). An example of a complete \Sapl program is
given here below:

\begin{CleanCode}
::boolean    = True | False
::list       = Nil  | Cons x xs
 
ones         = let os = Cons 1 os
               in os 
f a b        = let c  = add a 1,
                   d  = add b 2
               in add c d
fac n        = if (eq n 0) 1
                           (mult n
                                 (fac (sub n 1)))
filter f xxs = select xxs Nil
                          (\x xs = if (f x) (Cons x
                                                  (filter f xs))
                                            (filter f xs))
\end{CleanCode}

%Note that with the above definition of boolean we can use \texttt{select} instead of \texttt{if} in \Sapl. 

%We will not give a full definition of \Sapl, but only stipulate it in characteristics.
%he body of 
%\Sapl is described by the following abstract syntax definition:
%\begin{haskell}
%function    &::= identifier \{identifier\}* '\hspace{-1.5mm}=' expr\\
%expr        &::= application | \ '\lambda' \{identifier\}+ '\hspace{-1.5mm}=\hspace{-0.6mm}' expr\\
%application &::= factor \{factor\}*\\
%factor      &::= identifier | integer | \ '(' expr ')'
%\end{haskell}
%The following EBNF describes the syntax of \Sapl.
%\\ VOLLEDIGE EBNF OF ALLEEN EEN PAAR VOORBEELDEN



