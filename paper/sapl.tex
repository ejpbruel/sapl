\section{The \Sapl Programming Language and Interpreter}\label{sapljs:sec:sapl}
\Sapl stands for \textbf{S}imple \textbf{A}pplication \textbf{P}rogramming \textbf{L}anguage. The 
basic version of \Sapl uses function application as only operation. To make this possible, algebraic 
data types are implemented as functions. Details about how this is done can be found in \cite{JKP}.
Here we only give a few examples.
We start with the encoding of the list data type using functions, together with the \emph{sum} function.
\begin{haskell}
Nil f g      &=  f\\
Cons x xs f g &= g x xs\\
sum ys &= select ys 0   (\lambda x xs = x + sum xs)\\
\end{haskell}\normalsize
The \emph{select} operator semantically acts as the identity function and is only put there to enable a 
more efficient implementation of the \Sapl interpreter (again see \cite{JKP}).
The different cases in the \emph{select} are now functions that act on the arguments of a constructor.
As a more complex example consider a pattern based \Haskell function like \emph{mappair}.  
\begin{haskell}
mappair f Nil         &\ zs           &= Nil \\
mappair f (Cons x xs) &\ Nil          &= Nil \\
mappair f (Cons x xs) &\ (Cons y ys)  &= Cons (f x y) (mappair f xs ys) 
\end{haskell}\normalsize
This definition can be transformed to the following \Sapl function (using the above definitions of \emph{Nil} and \emph{Cons}).
\begin{haskell}
mappair f as zs \\
= select as  &Nil  \\
	         & (\lambda x xs = select  &\ zs  \\
				& & Nil  \\
				& & (\lambda y ys = Cons (f x y) (mappair f xs ys)))\\
\end{haskell}\normalsize 
%
\Sapl is used as an intermediate formalism for the interpretation of functional programming languages 
like \Haskell and \Clean.  For \Clean we included a \Sapl back-end in the \Clean compiler that 
generates \Sapl code that is ready to run in the \Sapl interpreter. 
For the \Sapl interpreter versions both implemented in \Java and \textsf{C} exist. The \Java
version can be loaded as an \Java Applet plug-in into web-browsers 
and can be used for execution of \iTask tasks at the client side.

\subsection{Some remarks on the definition of \Sapl}
\Sapl is very similar to the core languages of \Haskell and \Clean. 
Therefore, we will not give a full definition of \Sapl, but only say something about it's main characteristics.
Only constant let expressions are allowed (that may be cyclic).
$\lambda$ expressions may only occur as arguments of a \textsf{select}. We stipulate this by using a $'='$ instead of a $'\to'$.
%We will not give a full definition of \Sapl, but only stipulate it in characteristics.
%he body of 
%\Sapl is described by the following abstract syntax definition:
%\begin{haskell}
%function    &::= identifier \{identifier\}* '\hspace{-1.5mm}=' expr\\
%expr        &::= application | \ '\lambda' \{identifier\}+ '\hspace{-1.5mm}=\hspace{-0.6mm}' expr\\
%application &::= factor \{factor\}*\\
%factor      &::= identifier | integer | \ '(' expr ')'
%\end{haskell}
%The following EBNF describes the syntax of \Sapl.
%\\ VOLLEDIGE EBNF OF ALLEEN EEN PAAR VOORBEELDEN

\subsection{Co-operation between the Server and Client}
A special feature of the \Sapl interpreter is that we can use a dedicated form of \textsf{Clean} 
dynamics \cite{weea07n:PhD} for it. With dynamics it is possible to serialize a \textsf{Clean} expression 
(closure) to a string, store the string somewhere, retrieve the string at a later moment, turn it into a 
\textsf{Clean} expression again and execute it. We extended the dynamics features of \textsf{Clean} in 
such a way that it is also possible to serialize an expression in a \textsf{Clean} executable and de-
serialize it in the \Sapl interpreter (running the corresponding \Sapl program), and execute the 
expression there. This is a powerful feature because it makes it possible to migrate execution of a 
\textsf{Clean} program from server to client.  
This feature makes it possible the decide dynamically at run-time where to execute tasks:
at the server or at the client side.
We also used this feature to do client-side event handling for interactive (graphical) web-applications 
\cite{iEditors}. 


