\section{Introduction}\label{sapljs:sec:intro}
Client-side processing for  web applications is an important subject for research nowadays.
Non-strict purely functional languages like \Haskell \cite{Haskell} and \Clean \cite{Clean}
have many interesting properties, but their use on the client side of real world web-applications 
has so far been limited. 
This is at least partly due to the lack of browser support for these languages.
Nowadays a significant part of software development takes place in the browser, mostly 
using \JavaScript. 
Therefore, the availability of an implementation for non-strict functional languages 
in the browser has the potential to significantly 
improve the applicability of these languages in this area.
Several implementations of non-strict purely functional languages in the browser already exist. 
However, these implementations are either based on the use of \Java Applets 
(e.g. for \Sapl, a client-side \Clean platform  \cite{JKP,PJKA} )
or on the use of a dedicated plug-in 
(e.g. for \textsf{HaskellScript} \cite{HaskellScript} a \Haskell like functional language).
Both require the loading of a plug-in, which is often infeasible in environments where the user has no 
control over the 
configuration of his/her system. As an alternative to this, one might consider the use of \JavaScript. 
A \JavaScript interpreter is included with every major browser, so that installing a plug-in would no longer be required. 
Although traditionally perceived as being slower than languages like \Java and \C, the introduction of JIT compilers for \JavaScript has changed this picture significantly. 
Modern implementations of \JavaScript, such as the V8 engine that ships with Google's Chrome browser, offer performance that often rivals that of \Java.
\JavaScript has been used as a target platform for the client-side implementation of other functional languages like HOP and LINKS \cite{HOP1,HOP2,LINKS1}. But these are strict functional languages, which simplifies the translation to \JavaScript considerably.

It is possible to translate the \Java Applet implementation of the \Sapl interpreter to \JavaScript 
using the Google Webtoolkit. 
However, for this particular case, this is a naive solution. \JavaScript offers many features not available in Java that provide opportunities for a more efficient implementation. One of these is the fact that \JavaScript, unlike \Java, is a dynamic language, allowing for solutions that are based on compilation rather than interpretation. This paper describes an implementation of the non-strict purely functional language \Sapl in \JavaScript, that is written with these features in mind.

We implemented a compiler that translates \Sapl to \JS functions. 
The implementation is based on representing unevaluated function applications (chunks) by \JS arrays 
and the just-in-time conversion 
into a \JS function call of a chunk by a dedicated \textsf{eval} function.
We present two versions of our compilation scheme. 
The first one represents a straightforward conversion of \Sapl to \JS functions.
In the second version a number of optimization are added.
The most important optimization is the use of a compile time analysis to avoid the creation 
of unnecessary chunks. 
Another optimization is in-lining of arithmetic operations.
The final results show that it is indeed possible to realize a \Sapl to \JS conversion with code running at a speed that is competitive with that
of the \Sapl interpreter itself and that of other interpreters.
Summarizing we obtained the following results:
\begin{itemize}
\item The realization of a client-side implementation platform for the non-strict functional programming
language \Clean without the use of a plug-in.
\item This platform offers a performance that is competitive with that of the \Sapl interpreter
and other interpreters for non-strict functional languages.
\item The translation scheme is straightforward, using a one-to-one mapping of \Sapl onto \JS
functions.
\item The encoding relies on a simple representation of chunks by \JS arrays and the use of a \texttt{eval} function to turn chunks in function applications at the moment the result is needed.
\item The generated code is compatible with \JS in a sense that primitive data types 
are represented in the same way. 
Generated functions share the same name-space as those defined in native \JS.
Because of this generated code can interact with  \JS libraries.
\end{itemize}
The structure of the remainder of this paper is as follows:
We start with a motivation for this work in Section \ref{sapljs:sec:motivation}.
Section \ref{sapljs:sec:sapl} introduces \Sapl, the intermediate language we want to run on top of \JS.
The translation scheme of \Sapl to \JS is presented in Section \ref{sapljs:sec:sapljs}.
A number of benchmark tests for this implementation are presented in Section \ref{sapljs:sec:benchmarks}.
In Section \ref{sapljs:sec:optimizations} we propose a number of candidate optimizations that will be implemented in the near future.
We end with conclusion and look forward to future work in Section \ref{sapljs:sec:conclusions}.

%The representations that are the result of this translation rely on the use of an evaluator function eval to implement non-strict semantics.
%The eval function is used to reduce thunks (closures) on the moment they are needed. Using the dynamic compilation features of \JavaScript closures are turned into \JavaScript expressions that can be further executed.