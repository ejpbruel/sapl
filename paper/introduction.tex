\section{Introduction}\label{sapljs:sec:intro}
Client-side processing for web applications has become an important research
subject. Non-strict purely functional languages such as \Haskell and \Clean have
many interesting properties, but their use in client-side processing has been
limited so far. This is at least partly due to the lack of browser support for
these languages. Therefore, the availability of  an implementation for
non-strict purely functional languages in the browser has the potential to
significantly improve the applicability of these languages in this area.

Several implementations of non-strict purely functional languages in the browser
already exist. However, these implementations are either based on the use of a
\Java Applet (e.g. for \Sapl, a client-side platform for \Clean
\cite{JKP, PJKA}) or a dedicated plug-in (e.g. for \textsf{HaskellScript} 
\cite{HaskellScript} a \Haskell-like functional language). Both these solutions
require the installation of a plug-in, which is often infeasible in environments
where the user has no control over the configuration of his/her system.

As an alternative solution, one might consider the use of \JavaScript. A
\JavaScript interpreter is shipped with every major browser, so that the
installation of a plug-in would no longer be required.  Although traditionally
perceived as being slower than languages such as \Java and \C, the introduction 
of JIT compilers for \JavaScript has changed this picture significantly. Modern
implementations of \JavaScript, such as the V8 engine that is shipped with the
Google Chrome browser, offer performance that often rivals that of \Java.

\JavaScript has been used as the target language for the implementation of
other functional languages in the browser such as HOP and LINKS
\cite{HOP1, HOP2, LINKS1}. However, contrary to \Haskell and \Clean, these are
strict functional languages, which simplifies their translation to \JavaScript
considerably.

One possible strategy for implementing a functional language in the browser
using JavaScript is to use the Google Webtoolkit to translate the \Java Applet
implementation of the \Sapl interpreter to \JavaScript. However, for this
particular case, this is a naive approach: \JavaScript provides several
features, not available in \Java, that provide opportunities for a more
efficient implementation. Among these is the fact that, unlike \Java,
\JavaScript, is a {\em dynamic} language, allowing for solutions that are based
on compilation rather than interpretation. This paper describes an
implementation of the non-strict purely functional language \Sapl in
\JavaScript, written with these features in mind.

We implemented a compiler that translates \Sapl to \JS expressions. Its
implementation is based on the representation of unevaluated expressions
(thunks) as \JS arrays and the just-in-time evaluation of these thunks by a
dedicated \texttt{eval} function (it should be noted that this \texttt{eval}
function is different from the one that is provided by \JS itself).

We will present the translation scheme used by our compiler in two steps.
In step one, we describe a straightforward translation of \Sapl to \JS
expressions. In step two, we add several optimizations to the translation
scheme described in step one. Among these are the inlining of arithmetic
operations and the elimination of unnecessary thunks. Our final results show
that it is indeed possible to realize this translation scheme in such a way that
the resulting code runs at a speed competitive with that of the \Sapl
interpreter itself and that of other interpreters. Summarizing, we obtained the
following results:

\begin{itemize}
\item We realized an implementation of the non-strict purely functional
      programming language \Clean in the browser, via the intermediate language       \Sapl, that does not require the installation of a plug-in.
\item This performance of this implementation is competitive with that of the
      \Sapl interpreter and that of other interpreters for non-strict purely
      functional languages.
\item The underlying translation scheme is straightforward, constituting a
      one-to-one mapping of \Sapl onto \JS expressions.
\item The implementation of the compiler is based on the representation of
      unevaluated expressions as \JS arrays and the just-in-time evaluation of
      these thunks by a dedicated \texttt{eval} function.
\item The generated code is compatible with \JS in the sense that the namespace
      for functions is shared with that of \JS. This allows generated code to
      interact with \JS libraries.
\end{itemize}
The structure of the remainder of this paper is as follows: we start with a
motivation for our work in section \ref{sapljs:sec:motivation}. Section
\ref{sapljs:sec:sapl} introduces \Sapl, the intermediate language we intend to
implement in \JS. The translation scheme underlying this implementation is
presented in section \ref{sapljs:sec:sapljs}. Secton \ref{sapljs:sec:benchmarks}
presents a number of benchmark tests for the implementation. A number of
potential optimizations, to be implemented in the near future, is presented in
section \ref{sapljs:sec:optimizations}. Section \ref{sapljs:sec:relatedwork}
compares our approach with those used in earlier works. Finally, we end with our
conclusions and a summary of planned future work in section
\ref{sapljs:sec:conclusions}.

