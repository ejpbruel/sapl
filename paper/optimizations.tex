\section{Optimizations}\label{sapljs:sec:optimizations}
Above we described a straightforward compilation scheme for \Sapl to \JS, where function calls (closures) are transformed to arrays.
The only optimization we already made is replacing arithmetic expressions directly by there \JS counterparts 
and thereby forcing the evaluation of there arguments. 
In this way we may impose unwanted eager sub-expressions into a program. 
If a programmer does not want this he or she should use an indirection via an own-made auxiliary function.

The \textsf{eval} functions is used to turn a closure into a real \JS function call and reduces a closure to head-normal-form.
The \textsf{eval} functions has to do a case analysis on the structure of the closure expression.
A closure expression can either be a primitive value (integer, boolean, string), a boxed value, a constructor value or a real function application closure.
For the last case we measured that a direct \JS call is about 15 times faster than making the same call using the 
\textsf{Sapl.eval} function on the closure representing the call. This overhead is significant. 
Fortunately, in many cases we can do an analysis at compile time and replace a call of \textsf{eval} for a closure by the corresponding hard coded \JS.
This transformation replaces:
\begin{verbatim}
Sapl.eval([f,[a1,..,an])
\end{verbatim}
by:
\begin{verbatim}
f(a1,..,an)
\end{verbatim}
This may only be done if \textsf{f} is a known function (thus not a variable) and the number of applied arguments matches the
number of arguments of \textsf{f}.
This can be done at every place where an explicit \textsf{eval} call is done:
\begin{itemize}
\item The first argument of a \textsf{select} or \textsf{if}.
\item The arguments of an arithmetic operation.
\end{itemize}
But also at every place where a result is returned, because \textsf{eval} is called for this result immediately.


Examples

 

