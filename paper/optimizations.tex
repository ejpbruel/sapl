\section{Alternative Optimizations}
\label{sapljs:sec:optimizations}

Using the translation scheme introduced in section \ref{sapljs}, the following 
definition of the Fibonacci function, \texttt{fib}, in \Sapl:
\begin{CleanCode}
fib n = if (gt 2 n) 1 (add (fib (sub n 1)) (fib (sub n 2)))
\end{CleanCode} 
is translated to the following definition in \JS:
\begin{CleanCode}
function fib(n) {
    return n < 2 ? 1 : fib(eval(n) - 1) + fib(eval(n) - 2)
}
\end{CleanCode}
A simple strictness analysis should reveal that this definition can be turned into:
\begin{CleanCode}
function fib(n) {
    return n < 2 ? 1 : fib(n - 1) + fib(n - 2)
}
\end{CleanCode}
The calls to \texttt{eval} are now gone, which results in a huge improvement in performance (3 times as fast). Indeed,
this is how \texttt{fib} would have been written, had it been defined in \JavaScript
directly. In this particular example, the use of eager evaluation did not affect
the semantics of the function. However, this is not true in general. 

The \Clean to \Sapl compiler generates the strictness information needed to make these 
optimizations. The implementation of this requires a number of adaptations in the translation rules
and will be realized in the near future.

