\section{Why Client-side processing for Functional Programming Languages?}
\label{sapljs:sec:motivation}
Modern web applications use client-side processing to avoid round trips from the
client to the server in cases where transmission time dominates processing time 
for a request. Most web developers use different programming formalisms to
realize the client and server parts of an application. On the server-side,
languages such as \textsf{C++}, \Java, and \textsf{PHP} dominate the scene.
On the client-side, the use of \JS is the most obvious choice, due to the fact
that a \JS interpreter is shipped with every major browser. As an additional
advantage, browsers that support \JS usually also expose their \textsf{HTML}
\textsf{DOM} through a \JS API. This allows for the association of \JS functions to
\textsf{HTML} elements through the use of event listeners, and the use of \JS
functions to manipulate these same elements. This notwithstanding, the use of
multiple formalisms complicates the development of Internet applications
considerably, due to the close collaboration required between the client and
server parts of most web applications.

Several attempts have been made to overcome the above problem. An example of them
is the Google Web Toolkit (\textsf{GWT}). Using the GWT, server and client
code can be generated from the same \Java code base. This is accomplished
by translating the client part of the application to \JS, using special
libraries to ensure that client and server collaborate in the right way.

Another approach, based on the use of a non-strict purely functional programming
language, is the \iTask system \cite{ITASK}. \iTask is a combinator library
that is written in \Clean, and is used for the realization of web-based dynamic
workflow systems. % \cite{LDTA2010}. 
An \iTask application consists of a structured
collection of tasks to be performed by users, computers or both. \iTask
specifications allow the flow of control and information between tasks to be
expressed. 

To enhance the performance of \iTask applications, the possibility to handle
tasks on the client was added \cite{ITASK_AJAX}, accomplished by the
addition of a simple \texttt{OnClient} annotation to a task. When this
annotation is present, the \iTask runtime automatically takes care of all
communication between the client and server parts of the application. 
The client part is executed by the \Sapl interpreter, 
which is available as a \Java applet on the client.

\subsection{Why switch to \JS?}
Until now we used the \Java Applet implementation of the \Sapl interpreter to
run \Clean programs on the client side of web applications. However, this
approach has several drawbacks. First of all, the use of \Java Applets requires
the \Java Virtual Machine (\textsf{JVM}) to be installed, which is often infeasible in
environments where the user has no control over the configuration of his/her
system. Secondly, the \textsf{JVM} exhibits significant latency during start-up. And
third, the \textsf{JVM} might not even be available on certain platforms (on mobile
devices in particular).

In contrast to the \textsf{JVM}, a \JavaScript virtual machine is shipped with all major 
browsers for most major platforms (including mobile devices). Over the past few
years client-side processing has become more and more important, and as a result
much effort is spent on improving the speed of these \JS interpreters. One
significant improvement that has recently been made is the addition of
just-in-time \textsf{JIT} compilation techniques to modern \JS interpreters such
as Google V8 (which is used in the Google Chrome browser). The resulting gains
in speed have made \JS an interesting alternative for the implementation of the
\Sapl interpreter.

\subsection{Co-operation between the Server and Client}
A special feature of the \Sapl interpreter is that we can use a dedicated form
of \textsf{Clean} dynamics \cite{DYNAMICS} for it. This allows \Clean expressions (e.g.
partial function applications) to be serialized to strings. These strings
can subsequently be stored somewhere, and at a later moment be retrieved,
deserialized and executed.

We extended the dynamics features of \Clean in  such a way that it is also
possible to serialize an expression in a \Clean executable, deserialize it
in the \Sapl interpreter (running the corresponding \Sapl program), and execute
the expression there. This is allows the execution of a \Clean program to be
migrated from the server to the client, and thus to decide at run-time whether
to execute a task on the server or the client. We used this feature to
implement client-side event handling for interactive web applications.
\cite{iEditors}. 
