\section{A \JS based implementation for  \Sapl}\label{sapljs:sec:sapljs}
In Section \ref{sapljs:sec:motivation} we already motivated our choice for \JS as an implementation platform for \Sapl.
Of course, it is possible to build a \Sapl interpreter in \JS in an equivalent way to that of the \Java version.
But \JS offers a number of features that make it possible to do it in another and better way.
First, it is possible to add \JS code to a running \JS program on-the-fly by using the \textsf{eval} functions.
Second, it is possible to build function calls to \JS function dynamically by supply a function name and an array of arguments.
Third, \JS has no static type check. Normally we consider this as a serous drawback, but here we can use this to build heterogeneous arrays,
containing elements of different types.

Therefore, we have chosen to build a compiler that translates \Sapl code to \JS and that exploits the above mentioned possibilities.
In our translation scheme we map a \Sapl function onto a \JS function with the same name and number of arguments.
The most important issue in the translation is the representation of closures and constructors.
We decided to represent a closure: $f a_0 .. a_n$ by a \JS array: $[f,[a_0, .. ,a_n]]$.
A constructor:  $C a_0 .. a_n$ by a \JS array: $[k,[C,a_0, .. ,a_n]]$, where $k$ is a number that is equal to the 
to place in the original type definition of the constructor (see also below).
The translation scheme is as follows:

\begin{haskell}
T\llbracket f a_0 .. a_n = body\rrbracket  &\equiv &\ function f(a_0, .. ,a_n) = T\llbracket body \rrbracket \\ 
T\llbracket f a_0 .. a_n \rrbracket &\equiv  &\ [f, [T \llbracket a_0\rrbracket , .. ,T \llbracket a_n ] ]\\
T\llbracket ::t = C_0  args_0  |   &\equiv  &\    function C_0(args_0){return [0,[C_0,args_0]];} .. \\
\hspace{1.15cm}.. | C_0  args_0\rrbracket   & &\ function C_n(args_n){return [n,[C_n,args_n]];} \\ 
T\llbracket select t a_0 .. a_n\rrbracket &\equiv  &\ var rr = eval(t); switch(eval(t)) {case 0: }
\end{haskell}

\begin{align*}
T \llbracket k \rrbracket &\equiv k
\end{align*}

\begin{align*}
T \llbracket v \rrbracket &\equiv v
\end{align*}

\begin{align*}
T \left \llbracket \begin{array}{cccccc}
let(rec) & v_1    & = & B_1    &    & \\
         & \ldots & = & \ldots &    & \\
         & v_n    & = & B_n    & in & E
\end{array} \right \rrbracket &\equiv \begin{array}{l}
function\ (v_1, \ldots, v_n)\ \{ \\
\ \ \ \ var\ x_1 = \llbracket B_1 \rrbracket; \\
\ \ \ \ \ldots \\
\ \ \ \ var\ x_n = \llbracket B_n \rrbracket; \\
\\
\ \ \ \ return\ \llbracket E \rrbracket; \\
\}\ ()
\end{array}
\end{align*}

\begin{align*}
T\llbracket \lambda v_1 \: \ldots \: v_n \rightarrow E \rrbracket &= \begin{array}{l}
function\ (v_1, \ldots, v_n)\ \{ \\
\ \ \ \ return\ \llbracket E \rrbracket; \\
\}
\end{array}
\end{align*}

\begin{align*}
T \llbracket E_1 \: E_2 \: \ldots \: E_n \rrbracket &
    \equiv [T \llbracket E_1 \rrbracket, T \llbracket E_2 \rrbracket, \ldots, T \llbracket E_n \rrbracket]]
\end{align*}

\begin{align*}
T \llbracket op_1 \: E_1 \rrbracket &
        \equiv op_1 \: eval(T \llbracket E_1 \rrbracket) \\
T \llbracket E_1 \: op_2 \: E_2 \rrbracket &
        \equiv eval(T \llbracket E_1 \rrbracket \: op_2 \: eval(T \llbracket E_2 \rrbracket) \\
T \llbracket E_1 \: ? \: E_2 : E_3 \rrbracket &
        \equiv eval(T \llbracket E_1 \rrbracket) \: ? \: eval(T \llbracket E_2 \rrbracket) : eval(T \llbracket E_3 \rrbracket)
\end{align*}
%
Example translations:
\begin{CleanCode}
:: list = nil | cons x xs	
filter f xs = select xs nil (\a as = if (f a) (cons a (filter f as)) (filter f as))
\end{CleanCode}
%
Is translated to:
\begin{CleanCode}
function nil() {return [0, ['nil']];}
function cons(x, xs) {return [1, ['cons', x, xs]];}

function filter(f, xs) {
    var ys = Sapl.eval(xs);
    switch (ys[0]) {
    case 0:
        return nil();
    case 1:
        var a = ys[1][1],
            as = ys[1][2];
        return (Sapl.eval([f, [a]])) ? (cons(a, [filter, [f, as]])) : 
                                       (Sapl.eval([filter, [f, as]]));
    }
}\end{CleanCode}
The example shows how the structure of the function is preserved in the translation.
The \emph{eval} function is used to evaluate closures.\\
- it turns closures into function calls\\
- it writes the result at the first position of the closure to enable sharing\\
- primitive values are return ned unchanged\\
- a boxed value is unboxed\\
- built in functions on numbers, boolean, string, etc are executed by evaluating their arguments and applying the operation to them

\subsection{Normal Forms}
Above we described a straightforward compilation scheme for \Sapl to \JS, where function calls (closures) are transformed to arrays.
The only optimization we already made is replacing arithmetic expressions directly by there \JS counterparts 
and thereby forcing the evaluation of there arguments. 
In this way we may impose unwanted eager sub-expressions into a program. 
If a programmer does not want this he or she should use an indirection via an own-made auxiliary function.

The \textsf{eval} functions is used to turn a closure into a real \JS function call and reduces a closure to head-normal-form.
The \textsf{eval} functions has to do a case analysis on the structure of the closure expression.
A closure expression can either be a primitive value (integer, boolean, string), a boxed value, a constructor value or a real function application closure.
For the last case we measured that a direct \JS call is about 15 times faster than making the same call using the 
\textsf{Sapl.eval} function on the closure representing the call. This overhead is significant. 
Fortunately, in many cases we can do an analysis at compile time and replace a call of \textsf{eval} for a closure by the corresponding hard coded \JS.
This transformation replaces:
\begin{verbatim}
Sapl.eval([f,[a1,..,an])
\end{verbatim}
by:
\begin{verbatim}
f(a1,..,an)
\end{verbatim}
This may only be done if \textsf{f} is a known function (thus not a variable) and the number of applied arguments matches the
number of arguments of \textsf{f}.
This can be done at every place where an explicit \textsf{eval} call is done:
\begin{itemize}
\item The first argument of a \textsf{select} or \textsf{if}.
\item The arguments of an arithmetic operation.
\end{itemize}
But also at every place where a result is returned, because \textsf{eval} is called for this result immediately.

<<start of Eddy's part>>\\
A major source of overhead in the current implementation are the calls to the
evaluator function {\texttt eval}, generated by applying the translation scheme
$T$ to built-in operators. To illustrate this, consider the following definition
of the fibonacci function, $fib$, in SAPL:

\begin{CleanCode}
fib = \ n -> n < 2 ? 1 : fib (n - 1) + fib (n - 2)
\end{CleanCode}

Applying $T$ to this expression yields the following code in JavaScript:

\begin{CleanCode}
var fib = function (n) {
    return n < 2 ? 1 : eval([fib, [eval(n) - 1]]) + eval([fib, [eval(n) - 2]]);
}
\end{CleanCode}

Using this definition, the number of calls to {\texttt eval} is proportional to
$O(n^2)$, which is prohibitive even for small values of $n$.


In the above example, two thunks are created for the recursive applications of
$fib$, only to be immediately reduced again to normal form because their value
is required by the operator $+$. It should be obvious that these applications
might as well be performed immediately in this case, thus avoiding both the
construction of the thunks and the outer two calls to {\texttt eval}. In
general, $T$ generates a call to {\texttt eval} when an expression is required
to be in normal form (for instance because it is used as an operand to a
built-in operator). However, if it is known at compile time that an expression
already is in normal form, the call to {\texttt eval} is unnecessary, and can
be avoided.

To take advantage of this observation, a new translation scheme, $S$, is
introduced, which is equivalent to $T$, except for the added restriction that
expressions generated by $S$ should always be in normal form. To adhere to this
restriction, the rule for function applications has to be rewritten as follows:

\begin{equation*}
S \llbracket E_1 \: E_2 \: \ldots \: E_n \rrbracket \equiv
\begin{cases}
[T \llbracket E_1 \rrbracket, [T \llbracket E_2 \rrbracket,
                               \ldots,
                               S \llbracket E_n \rrbracket]] &
        \text{if $n \le m$} \\
T \llbracket E_1 \rrbracket(T \llbracket E_2 \rrbracket,
                            \ldots,
                            T \llbracket E_n \rrbracket) &
        \text{if $n = m$} \\
eval([ T \llbracket E_1 \rrbracket(T \llbracket E_2 \rrbracket,
                                   \ldots,
                                   T \llbracket E_n \rrbracket),
                        [T \llbracket E_{m+1} \rrbracket,
                         \ldots,
                         T \llbracket E_n \rrbracket]]) &
        \text{otherwise (if $n > m$)}
\end{cases}
\end{equation*}
where $m$ is the arity of $E_1$.

The above rule assumes that $E_1$ is a lambda abstraction, and that the arity of
$E_1$ is known. The possibility that $E_1$ is actually an identifier introduces
the need for a symbol table, which allows names to be resolved to their
corresponding bindings. Furthermore, it is impossible to determine the arity of
a function at compile time in the following to the cases:

\begin{enumerate}
\item The function was passed as an argument to the current function
\item The function was returned as a the result of a function application
\end{enumerate}

In both cases, a call to {\texttt eval} is still necessary. The former case
occurs when $n > m$, and is already handled in the above rule. The latter case
occurs when $E_1$ is resolved to be a local identifier, and is handled by the
following additional rule:

\[ S \llbracket E_1 \: E_2 \: \ldots \: E_n \rrbracket \equiv
   eval(T \llbracket E_1 \: E_2 \: \ldots \: E_n \rrbracket) \]

Note how the behavior of $S$ closely mimics that of {\texttt eval}. Indeed,
from a conceptual point of view, $S$ performs the call to {\texttt eval} at
compile time, rather than run time. This allows the rule in $T$ for translating
built-in operators to be rewritten as follows:

\begin{align*}
T \llbracket op_1 \: E_1 \rrbracket &
        \equiv op_1 \: S \llbracket E_1 \rrbracket \\
T \llbracket E_1 \: op_2 \: E_2 \rrbracket &
        \equiv S \llbracket E_1 \rrbracket \: op_2 \:
               S \llbracket E_2 \rrbracket \\
T \llbracket E_1 \: ? \: E_2 \: : \: \rrbracket &
        \equiv S \llbracket E_1 \rrbracket \: ? \:
               S \llbracket E_2 \rrbracket \: : \:
               S \llbracket E_3 \rrbracket 
\end{align*}

Observe that this new rule does not generate any calls to {\texttt eval}, and
that $S$ only generates calls to {\texttt eval} if the arity of a function
cannot be determined at compile time.

In the above discussion, it was silently assumed that the result of a function
application is always in head normal form. To meet this requirement, however,
the rules for translating let(rec)-expressions and functions have to be
rewritten such that the expression returned is translated using $S$, rather than
$T$.



