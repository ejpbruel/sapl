\section{A \JS based implementation for  \Sapl}
\label{sapljs:sec:sapljs}
Section \ref{sapljs:sec:motivation} motivated our choice for implementing the
\Sapl interpreter in the browser using \JS. Compared to \Java, this language 
provides several features that offer opportunities for a more efficient 
implementation. First of all, the fact that \JS is a \emph{dynamic} language allows both functions and function calls to be generated at run-time, using the 
built-in functions \texttt{eval} and \texttt{apply}, respectively. Second, the 
fact that \JS is a dynamically \emph{typed} language allows the creation of 
heterogeneous arrays. Therefore, rather than building an interpreter, we have 
chosen to build a compiler/interpreter hybrid that exploits the features 
mentioned above.

As mentioned in section \ref{sapljs:sec:sapl}, the original \Sapl interpreter 
provides no special constructs for algebraic data types. The \texttt{select}
keyword is used as a hint to the compiler that allows constructor based case selection to be optimized, but it is also possible to represent constructors 
as functions that act as case selectors themselves. As it turns out, the
translation to \JS can be made more efficient by representing constructors in a
special way. Therefore, it is no longer possible to represent constructors as ordinary functions, and special constructs for constructors have been added to
the language. Furthermore, the \texttt{select} keyword is now obligatory when doing constructor based case analysis.

For the following \Sapl constructs we must describe how they are translated to 
\JS:
\begin{itemize}
	\item constants, such as booleans, integers, real numbers, and strings;
    \item identifiers, such as variable and function names;
    \item let constructs;
	\item function definitions;
	\item applications;
	\item constructor definitions;
	\item select statements;
	\item if statements;
	\item built-in functions, such as \texttt{add}, \texttt{eq}, etc.
\end{itemize}

\subsubsection{Constants}
Constants do not have to be transformed. They have the same representation in
\Sapl and \JS.

\subsubsection{Identifiers}
Identifiers in \Sapl and \JS share the same namespace. Therefore, identifiers do
not have to be transformed either.

\subsubsection{\textsf{Let} constructs} 
\texttt{Let} constructs are translated differently depending on whether they
are cyclic or not. Non-cyclic lets in \Sapl can be translated to \texttt{var}
declarations in \JS, as follows: 
\begin{CleanCode}
let x = b in e
\end{CleanCode}
is translated to:
\begin{CleanCode}
var x = tb; te
\end{CleanCode}
where \texttt{tb} and \texttt{te} are the result of the translation of 
\texttt{b} and \texttt{e}, respectively.

Due to \JS's support for closures, cyclic lets can be translated from \Sapl to
\JS in a straightforward manner. The idea is to take any occurences of 
\texttt{x} in \texttt{b} and replace them with:
\begin{CleanCode}
function () { return b; }
\end{CleanCode}
This construction relies on the fact that \JS combines a function with any free
variables used by that function to form a closure.

\subsubsection{Function definitions} 
Due to \JS's support for higher-order functions, function definitions can be
translated from \Sapl to \JS in a straightforward manner:
\begin{CleanCode}
f x1 ... xn = body
\end{CleanCode}
is translated to:
\begin{CleanCode}
function f(x1, ..., xn) { tbody }
\end{CleanCode}
where \texttt{tbody} is the result of the translation of \texttt{body}.

\subsubsection{Applications} 
Every \Sapl expression is an application. Due to \JS's eager 
evaluation semantics, applications cannot be translated from \Sapl to 
\JS directly. Instead, unevaluated expressions (or \emph{thunks}) in \Sapl are translated to arrays in \JS:
\begin{CleanCode}
x0 x1 .. xn 
\end{CleanCode}
is translated to:
\begin{CleanCode}
[tx0, [tx1, ..., txn]]
\end{CleanCode}
where \texttt{txi} is the result of the translation of \texttt{xi}.

\subsubsection{Constructor definitions} 
Constructor definitions in \Sapl are translated to arrays in \JS, in such a way 
that they can be used in a \texttt{select} construct to select the right case. 
A \Sapl type definition containing constructors is translated as follows:
\begin{CleanCode}
:: typename = ... | Ck x0 ... xn | ...
\end{CleanCode}
is transformed to:
\begin{CleanCode}
... function Ck(x0, ..., xn) { return [k, [`Ck', tx0, ..., txn]]; } ...
\end{CleanCode}
where \texttt{k} is a positive integer, corresponding to the position of the 
constructor in the original type definition, 
and \texttt{txi} is the result of the translation of \texttt{xi}. 
The name of the constructor, \texttt{`Ck'}, is 
put into the result for printing purposes.

\subsubsection{\textsf{select} statements} 
A \texttt{select} statement in \Sapl is translated to a \texttt{switch} 
statement in \JS, as follows:
\begin{CleanCode}
select f (\x0 ... xn = b) ..
\end{CleanCode}
is translated to:
\begin{CleanCode}
var _tmp = Sapl.eval(tf);
switch(_tmp[0]) {
	case 0: var x0 = _tmp[1][1], ..., xn = _tmp[1][n+1];
	        tb;
            break;
	...
};
\end{CleanCode}
where \texttt{tf} and \texttt{tb} are the result of the translation of
\texttt{f} and \texttt{b}, respectively. 

Evaluating the first argument of a \texttt{select} statement yields an array representing a constructor (see above). The first argument in this array 
represents the position of the constructor in its type definition, and is used 
to select the right case in the definition. The parameters of the $\lambda$-
expression for each case are bound to the corresponding arguments of the constructor in the \texttt{var} declaration (see also examples).

\subsubsection{\textsf{if} statements} 
An \texttt{if} statement in \Sapl is translated to the conditional operator in 
\JS, as follows:
\begin{CleanCode}
if p t f
\end{CleanCode}
is translated to:
\begin{CleanCode}
Sapl.eval(tp) ? tt : tf;
\end{CleanCode}
where \texttt{tp}, \texttt{tt}, \texttt{tf} are the result of the translation of
\texttt{p}, \texttt{t}, \texttt{f}, respectively. This translation works because
booleans in \Sapl and \JS have the same representation.

\subsubsection{Built-in Functions}
\Sapl defines several built-in functions for arithmetic and logical operations.
As an example, the \texttt{add} function is defined as follows:
\begin{CleanCode}
function add(x, y) { return Sapl.eval(x) + Sapl.eval(y); }
\end{CleanCode}
Unlike user-defined functions, built-in functions such as \texttt{add} have
strict evaluation semantics. To guarantee that they are in normal form when the 
function is called, the function \texttt{Sapl.eval} is applied to its arguments
(see section \ref{eval}).

\subsection{Examples}
The following definitions in \Sapl:
\begin{CleanCode}
:: list = Nil | Cons x xs	

ones = let os = Cons 1 os in os
fac n = if (eq n 0) 1 (mult n (fac (sub n 1)))
sum xxs = select xxs 0 (\x xs = add x (sum xs))
\end{CleanCode}
%
are translated to the following definitions in \JS:
\begin{CleanCode}
function Nil() { return [0, ['nil']]; }
function Cons(x, xs) { return [1, ['Cons', x, xs]]; }

function ones() { var os = Cons(1, function() { return os; }); return os; }

function fac(n) {
    return (eq(n,0)) ? 1 : [mult, [n, [fac, [[sub, [n, 1]]]]]];
}

function sum(as) {
	var _tmp = Sapl.eval(as);
	switch (_tmp[0]) {
		case 0: return 0;
		case 1: var x = _tmp[1][1], xs = _tmp[1][2]; 
				return [add, [x, [sum, [xs]]]];
	};
}

\end{CleanCode}
The examples show that the translation is straightforward and preserves the
structure of the original definitions.

\subsection{The \texttt{eval} function}
\label{eval}
To emulate \Sapl's non-strict evaluation semantics for function applications, 
we represented unevaluated expressions (thunks) as arrays in \JS. Because \JS
treats these arrays as primitive values, some way is needed to explicitly reduce
thunks to normal form when their value is required. This is the purpose of the
\texttt{Sapl.eval} function. It performs a case analysis on an expression and
undertakes different actions based on its type:

\subsubsection{Constants}
If the expression is a constant or a constructor, it is returned immediately. 
Constants and constructors are already in normal form.

\subsubsection{Thunks}
If the expression is a thunk of the form \texttt{[f, [xs]]}, it is transformed 
into a function call \texttt{f(xs)}, and \texttt{Sapl.eval} is applied 
recursively to the result (this is necessary because the result of a function 
call may be another thunk).

Due to \JS's reference semantics for arrays, thunks may become shared between 
expressions over the course of evaluation. To prevent the same thunk from being 
reduced twice, the result of the call is written back into the array. If this
result is a primitive value, the array is transformed into a \emph{boxed value}
instead. Boxed values are represented as arrays of size one. Note that in \JS,
the size of an array can be altered in-place.

If the number of arguments in the thunk is smaller than the arity of the 
function, it cannot be further reduced (is already in normal form), so it is
returned immediately. Conversely, if the number of arguments in the thunk is
larger than the arity of the function, a new thunk is constructed from the
result of the call and the remainder of the arguments, and \texttt{Sapl.eval}
is applied recursively to the result.

\subsubsection{Boxed values}
If the expression is a boxed value of the form \texttt{[x]}, the value 
\texttt{x} is unboxed and returned immediately (only primitive values are boxed,
which are in normal form).

\subsubsection{Curried applications}
If the expression is a curried application of the form
\texttt{[[f, [xs]], [ys]]}, it is transformed into  \texttt{[f, [xs ++ ys]]}, 
and \texttt{Sapl.eval} is applied recursively to the result.

\subsection{Normal Forms}
Above we described a straightforward compilation scheme from \Sapl to \JS, 
where unevaluated expressions (thunks) are translated to arrays. The
\texttt{Sapl.eval} function is used to reduce thunks to normal form when their
value is required. For ordinary function calls, our measurements indicate that 
the use of \texttt{Sapl.eval} is more than 10 times slower than doing the same
call directly. This constitutes a significant overhead. Fortunately, a simple
compile time analysis reveals many opportunities to eliminating unnecessary 
thunks in favor of such direct calls. Thus, expressions of the form:
\begin{CleanCode}
Sapl.eval([f, [x1, ..., xn])
\end{CleanCode}
are substituted with:
\begin{CleanCode}
f(x1, ..., xn)
\end{CleanCode}
This substitution is only possible if \texttt{f} is a function with known arity
at compile-time, and the number of arguments in the thunk is equal to the arity
of the function. It can be performed wherever a call to \textsf{Sapl.eval} 
occurs:
\begin{itemize}
\item The first argument to a \texttt{select} or \texttt{if};
\item The arguments to a built-in function;
\item Expressions that follow a \texttt{return} statement in \JS. These
      expressions are always evaluated immediately after they are returned.
\end{itemize}

As an additional optimization, arithmetic operations are inlined wherever they
occur. With these optimizations added, the earlier definitions ot \texttt{sum} 
and \texttt{fac} are now translated to:
\begin{CleanCode}
function fac(n){
	return (Sapl.eval(n) == 0) ? 1 : Sapl.eval(n) * fac(Sapl.eval(n) - 1);
}
\end{CleanCode}
\begin{CleanCode}
function sum(xxs) {
	var _tmp = Sapl.eval(xxs);
	switch(_tmp[0]){
		case 0: return 0;
		case 1: var x = _tmp[1][1], xxs = _tmp[1][2]; 
		        return Sapl.eval(x) + sum(xs);
	}
}
\end{CleanCode}
As will be shown in the next section, the above optimizations result in
significant speed gains.
