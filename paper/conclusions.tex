\section{Conclusion and Future Work}\label{sapljs:sec:conclusions}
In this paper we evaluated the use of \JS as a target language for lazy functional programming languages like \Haskell of \Clean using the intermediate language \Sapl.
The implementation realized has the following characteristics:

\begin{itemize}
\item It achieves a speed for compiled benchmarks 
that is competitive with that of the  \Sapl interpreter and other interpreters like \textsf{Amanda}, 
\textsf{Helium}, \textsf{Hugs} and \textsf{GHCi}.
This despite the fact that \JS has a more than 3 times slower execution speed than \Java, which was used for the \Sapl interpreter.
\item The implementation supports the full \Clean language (but not all libraries are supported).
We tested the implementation against a large number of \Clean programs compiled with \Clean to \Sapl compiler. 
\item The runtime speed of benchmarks is often dominated by memory operations. 
But in many cases this overhead could be significantly reduced by a simple optimization
that reduces the creation thunks.
\item It is expected that the use of strictness information will give a further increase of performance
for many benchmarks.
%\item The implementation is straightforward based on a simple transformation scheme and using a few simple to implement optimizations.
%\item Functions from the source are in one-one correspondence with \JS functions.
%\item The lazy semantics is maintained by the encoding of thunks by \JS arrays and using a special \textsf{eval} function to turn thunks in function calls.
%\item Primitive data types like integer, string, boolean and char in the compiled program are equivalent to their \JS counterparts, which simplifies interactions with native \JS functions.
\end{itemize}

\subsection{Future Work}
We have planned the following future work:
\begin{itemize}
\item The realization of  an even better performance by making use of the strictness information generated by the \Clean compiler.
\item Automatic conversion of other data types like Records, Arrays, etc, between \Sapl and \JS.
\item The addition of dedicated libraries to access the DOM, etc.
\item The creation of special client-side \iTask libraries.
\item Further improvements of performance by making better compile time analysis (e.g. \JS currying instead of building thunks),
the use of tail recursion, etc.
\end{itemize}

\section*{Acknowledgements}
The authors would like to thank L\'aszl\'o Domoszlai for helping us with converting a preliminary version of the interpreter to one supporting the full \Sapl language and
for realizing a nice web-based user interface for the interpreter.
