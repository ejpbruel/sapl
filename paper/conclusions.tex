\section{Conclusion and Future Work}\label{sapljs:sec:conclusions}
In this paper we evaluated the use of \JS as a target language for lazy functional programming languages like \Haskell of \Clean using the intermediate language \Sapl.
We showed that we could achieve a speed for compiled benchmarks 
that is competitive with that of the  \Sapl interpreter and other interpreters like Amanda, Helium and GHCi.
We tested the implementation against a large number of Clean programs compiled with \Clean to \Sapl compiler. 
The implementation has the following characteristics:
\begin{itemize}
\item The implementation is straightforward based on a simple transformation scheme and using a few simple to implement optimizations.
\item The lazy semantics is maintained by the coding of closures by \JS arrays and using a special \textsf{eval} function to turn closures in function calls.
\item Functions from the source are in one-one correspondence with \JS functions.
\item Primitive data types like integer, string, boolean and char in the compiled program are equivalent to their \JS counterparts, which simplifies interactions with native \JS functions.
\end{itemize}

\subsection{Future Work}
We have planned to do the following future work:
\begin{itemize}
\item Automatic conversion of other data types like Records, ADT's, etc, between \Sapl and \JS.
\item The addition of dedicated libraries to access the DOM, etc.
\item Further improvement of performance by making better compile time analysis (e.g. \JS currying instead of making closures),
the use of tail recursion, etc.
\item Obtaining a better performance by making use of the strictness information the \Clean compiler generates.

\section*{Acknowledgements}
The authors would like to thank L\'aszl\'o Domoszlai for helping us at converting the original interpreter to one supporting the full \Sapl language and
for realizing a nice web-based user interface for the interpreter.
\end{itemize}