\section{Conclusion and Future Work}\label{sapljs:sec:conclusions}
In this paper we evaluated the use of \JS as a target language for lazy functional programming languages like \Haskell of \Clean using the intermediate language \Sapl.
Our main goal was to achieve an execution speed that is good enough to make this implementation of practical use. We showed that we could achieve a speed that is competitive with that of popular interpreters like the original \Sapl interpreter and that of Helium and GHCi.

The implementation has a number of important advantages. 
\begin{itemize}
\item Functions for the source language are in a one-one correspondence with functions in\JS.
\item Primitive data types like integer, string, boolean and char in the compiled program are equivalent to their \JS counter parts.
\item  
\item The generated program has access to 
\end{itemize}

Advantages: direct access to \JS functions. Special client libraries for Web pages manipulation.
Integration of record types arrays etc.
\subsection{Future Work}
\begin{itemize}
\item Automatic conversion of other data types like Records, ADT's, etc, between \Sapl and \JS.
\item The addition of dedicated libraries to access to DOM etc.
\item Further improvement of performance by making better compile time analysis (e.g. \JS currying instead of making closures).
\end{itemize}