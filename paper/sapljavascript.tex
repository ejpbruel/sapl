%%%%%%%%%%%%%%%%%%%%%%%%%%%%%%%%%%%%%%%%%%%%%%%%%%%%%%%%%%%%%%%%%%
%%                                                              %%
%%  IFL 2010 submission Eddy Bruel, JM Jansen  		 				      %%
%%                                                              %%
%%%%%%%%%%%%%%%%%%%%%%%%%%%%%%%%%%%%%%%%%%%%%%%%%%%%%%%%%%%%%%%%%%

\documentclass{llncs}
%\usepackage{makeidx}  % allows for indexgeneration
\usepackage{graphicx}  % allows for graphics

\usepackage{url}
\usepackage{Clean}
\usepackage{haskell}
\usepackage{xspace}
\usepackage{stmaryrd}
\usepackage{amsmath}
\newcommand{\Ajax}[0]{\mksf{Ajax}}
\newcommand{\Arrow}[0]{\mksf{Arrow}}
\newcommand{\bigwig}[0]{\mksf{$<$bigwig$>$}}
\newcommand{\bindd}[0]{\ensuremath{\sharp\!\!\!>\!\!\!>}}
\newcommand{\BPMN}[0]{\mksf{BPMN}}
\newcommand{\C}[0]{\mksf{C}}
\newcommand{\cgi}[0]{\mksf{CGI}}
\newcommand{\Clean}[0]{\mksf{Clean}}
\newcommand{\ConDec}[0]{\mksf{ConDec}}
\newcommand{\Curry}[0]{\mksf{Curry}}
\newcommand{\Declare}[0]{\mksf{\sc declare}}
\newcommand{\etal}[0]{\textit{et al}}
\newcommand{\flapjax}[0]{\mksf{Flapjax}}
\newcommand{\formlets}[0]{\mksf{formlets}}
\newcommand{\GEC}[0]{\mksf{GEC}}
\newcommand{\GenericHaskell}[0]{\mksf{GenericH$\forall$skell}}
\newcommand{\Google}[0]{\mksf{Google}}
\newcommand{\Haskell}[0]{\mksf{Haskell}}
\newcommand{\Hop}[0]{\mksf{Hop}}
\newcommand{\html}[0]{\mksf{HTML}}
\newcommand{\Html}[0]{\mksf{Html}}
\newcommand{\iData}[0]{\mksf{iData}}
\newcommand{\InternetExplorer}[0]{\mksf{Internet Explorer}}
\newcommand{\iTask}[0]{\mksf{iTask}}
\newcommand{\Java}[0]{\mksf{Java}}
\newcommand{\JavaScript}[0]{\mksf{JavaScript}}
\newcommand{\JS}[0]{\mksf{JavaScript}}
\newcommand{\Links}[0]{\mksf{Links}}
\newcommand{\LTL}[0]{\mksf{LTL}}
\newcommand{\Mawl}[0]{\mksf{Mawl}}
\newcommand{\mksf}[1]{{\textsf{#1}}\xspace}
\newcommand{\ML}[0]{\mksf{ML}}
\newcommand{\Mozilla}[0]{\mksf{Mozilla}}
\newcommand{\OCL}[0]{\mksf{OCL}}
\newcommand{\ObjectIO}[0]{\mksf{Object I/O}}
\newcommand{\ORM}[0]{\mksf{ORM}}
\newcommand{\PDM}[0]{\mksf{PDM}}
\newcommand{\PHP}[0]{\mksf{PHP}}
\newcommand{\php}[0]{\mksf{php}}
\newcommand{\Powerforms}[0]{\mksf{Powerforms}}
\newcommand{\Rails}[0]{\mksf{Rails}}
\newcommand{\Ruby}[0]{\mksf{Ruby}}
\newcommand{\Sapl}[0]{\mksf{Sapl}}
\newcommand{\Sapljs}[0]{\mksf{Sapljs}}
\newcommand{\Sparkle}[0]{\mksf{Sparkle}}
\newcommand{\SQL}[0]{\mksf{SQL}}
\newcommand{\UML}[0]{\mksf{UML}}
%\newcommand{\url}[1]{\mksf{#1}}
\newcommand{\uuxml}[0]{\mksf{UUXML}}
\newcommand{\WashCGI}[0]{\mksf{WASH/CGI}}
\newcommand{\WebDSL}[0]{\mksf{WebDSL}}
\newcommand{\WebWorkFlow}[0]{\mksf{WebWorkFlow}}
\newcommand{\wxWindows}[0]{\mksf{wxWindows}}
\newcommand{\xml}[0]{\mksf{XML}}

%\lstset{%
    language=Java,%
    basicstyle=\small,%\footnotesize,%\sffamily,%
    keywordstyle=\bfseries\ttfamily,%
    identifierstyle=\ttfamily,%
    stringstyle=\itshape\ttfamily,%
    commentstyle=\sffamily,%
%    commentstyle=\rmfamily,%
    showstringspaces=false,%
    basewidth=0.55em,%
%    frame=leftline,%
    framexleftmargin=-5mm,%
    tabsize=4,%
    literate= {->}{{$\rightarrow$}}2%
            {[]}{[$\ $]}2%
}

\lstnewenvironment{Java}{\lstset{language=Java}}{}
\newcommand{\prog}[1]{\lstinline[language=Java]�#1�}
\newcommand\CPP{C\kern -.12em \raise .6ex \hbox{\small{$+$}}\kern -.15em \raise .2ex \hbox{\small{$+$}}}


\begin{document}

\pagestyle{headings}  % switches on printing of running heads
\mainmatter              % start of the contributions

\title{Implementing a non-strict purely Functional Language in \JavaScript}
\titlerunning{Implementing a non-strict purely Functional Language in \JavaScript}
%
\author{Eddy Bru\"el\inst{1}, Jan Martin Jansen\inst{2}}
%
%
\authorrunning{Eddy Bru\"el, Jan Martin Jansen}
% abbreviated author list (for running head)
%
%%%% modified list of authors for the TOC (add the affiliations)
\tocauthor{Eddy Bru\"el, Jan Martin Jansen}
%
\institute{Vrije Universiteit Amsterdam \and Faculty of Military Sciences, \\Netherlands Defence Academy, Den 
Helder, the Netherlands
\email{ejpbruel@gmail.com,jm.jansen.04@nlda.nl}}

\maketitle              % typeset the title of the contribution

\begin{abstract}
This paper describes an implementation of a non-strict purely functional language in \JavaScript.
This particular implementation is based on the translation of a high-level functional language such as 
\Haskell or \Clean into \JavaScript via the intermediate functional language \Sapl. 
The resulting code relies on the use of an evaluator function to emulate the non-strict semantics of 
these languages.
The speed of execution is competitive with that of the \Sapl interpreter itself 
and other existing interpreters.
% like \textsf{Helium} and  \textsf{GHCi} and better than that of \textsf{Hugs}.
\end{abstract}

\section{Introduction}\label{sapljs:sec:intro}
Client-side processing for  web applications is an important subject for research nowadays.
Non-strict purely functional languages like \Haskell %\cite{Haskell} 
and \Clean %\cite{Clean}
have many interesting properties, but their use on the client side of real world web-applications 
has so far been limited. 
This is at least partly due to the lack of browser support for these languages.
Nowadays, a significant part of software development takes place in the browser, mostly 
using \JavaScript. 
Therefore, the availability of an implementation for non-strict functional languages 
in the browser has the potential to significantly 
improve the applicability of these languages in this area.
Several implementations of non-strict purely functional languages in the browser already exist. 
However, these implementations are either based on the use of \Java Applets 
(e.g. for \Sapl, a client-side \Clean platform  \cite{JKP,PJKA} )
or on the use of a dedicated plug-in 
(e.g. for \textsf{HaskellScript} \cite{HaskellScript} a \Haskell like functional language).
Both require the loading of a plug-in, which is often infeasible in environments where the user has no 
control over the 
configuration of his/her system. As an alternative to this, one might consider the use of \JavaScript. 
A \JavaScript interpreter is included with every major browser, so that installing a plug-in would no longer be required. 
Although traditionally perceived as being slower than languages like \Java and \C, the introduction of JIT compilers for \JavaScript has changed this picture significantly. 
Modern implementations of \JavaScript, such as the V8 engine that ships with Google's Chrome browser, offer performance that often rivals that of \Java.
\JavaScript has been used as a target platform for the client-side implementation of other functional languages like HOP and LINKS \cite{HOP1,HOP2,LINKS1}. But these are strict functional languages, which simplifies the translation to \JavaScript considerably.

It is possible to translate the \Java Applet implementation of the \Sapl interpreter to \JavaScript 
using the Google Webtoolkit. 
However, for this particular case, this is a naive solution. \JavaScript offers many features not available in Java that provide opportunities for a more efficient implementation. One of these is the fact that \JavaScript, unlike \Java, is a dynamic language, allowing for solutions that are based on compilation rather than interpretation. This paper describes an implementation of the non-strict purely functional language \Sapl in \JavaScript, that is written with these features in mind.

We implemented a compiler that translates \Sapl to \JS functions. 
The implementation is based on representing unevaluated expressions (chunks) by \JS arrays 
and the just-in-time conversion 
into a \JS expression of a chunk and the evaluation of it by a dedicated \textsf{eval} function.
We present two versions of our compilation scheme. 
The first one represents a straightforward conversion of \Sapl to \JS functions.
In the second version a number of optimization are added.
The most important optimization is the use of a compile time analysis to avoid the creation 
of unnecessary chunks. 
Another optimization is in-lining of arithmetic operations.
The final results show that it is indeed possible to realize a \Sapl to \JS conversion with code running at a speed that is competitive with that
of the \Sapl interpreter itself and that of other interpreters.
Summarizing we obtained the following results:
\begin{itemize}
\item The realization of a client-side implementation platform for the non-strict functional programming
language \Clean, via the intermediate language \Sapl, without the use of a plug-in.
\item This platform offers a performance that is competitive with that of the \Sapl interpreter
and other interpreters for non-strict functional languages.
\item The translation scheme is straightforward, using a one-to-one mapping of \Sapl onto \JS
functions.
\item The encoding relies on a simple representation of unevaluated expressions by \JS arrays 
and the use of an \texttt{eval} function to turn them back into expressions and to evaluate them at the moment the result is needed.
\item The generated code is compatible with \JS in a sense that primitive data types 
are represented in the same way. 
Generated functions share the same name-space as those defined in native \JS.
Because of this generated code can interact with  \JS libraries.
\end{itemize}
The structure of the remainder of this paper is as follows:
We start with a motivation for this work in Section \ref{sapljs:sec:motivation}.
Section \ref{sapljs:sec:sapl} introduces \Sapl, the intermediate language we want to run on top of \JS.
The translation scheme of \Sapl to \JS is presented in Section \ref{sapljs:sec:sapljs}.
A number of benchmark tests for this implementation are presented in Section \ref{sapljs:sec:benchmarks}.
In Section \ref{sapljs:sec:optimizations} we propose a number of candidate optimizations that will be implemented in the near future.
We end with conclusions and look forward to future work in Section \ref{sapljs:sec:conclusions}.

%The representations that are the result of this translation rely on the use of an evaluator function eval to implement non-strict semantics.
%The eval function is used to reduce thunks (closures) on the moment they are needed. Using the dynamic compilation features of \JavaScript closures are turned into \JavaScript expressions that can be further executed.
\section{Why Client-side processing for Functional Programming Languages?}
\label{sapljs:sec:motivation}
Modern web applications use client-side processing to avoid round trips from the
client to the server in cases where transmission time dominates processing time 
for a request. Most web developers use different programming formalisms to
realize the client and server parts of an application. On the server-side,
languages such as \textsf{C++}, \Java, and \textsf{PHP} dominate the scene.
On the client-side, the use of \JS is the most obvious choice, due to the fact
that a \JS interpreter is shipped with every major browser. As an additional
advantage, browsers that support \JS usually also expose their \textsf{HTML}
\textsf{DOM} through a \JS API, allowing for the association of \JS functions to
\textsf{HTML} elements through the use of event listeners, and the use of \JS
functions to manipulate these same elements. This not withstanding, the use of
multiple formalisms complicates the development of Internet applications
considerably, due to the the close collaboration required between the client and
server parts of most web applications.

Several attempts have been made to overcome the above problem. As an example we
present the Google Web Toolkit (\textsf{GWT}). Using the GWT, server and client
code can be generated from the same \Java code base. This is accomplished
by translating the client part of the application to \JS, using special
libraries to ensure that client and server collaborate in the right way.

Another approach, based on the use of a non-strict functional programming
language, is the \iTask system \cite{ITASK}. \iTask is a combinator library
that is written in \Clean, and used for the realization of web-based dynamic
workflow systems \cite{LDTA2010}. An \iTask application consists of a structured
collection of tasks to be performed by users, computers or both. \iTask
specifications allow the flow of control and information between tasks to be
expressed. 

To enhance the performance of \iTask applications, the possibility to handle
tasks on the client was added in \cite{ITASK_AJAX}, accomplished by the
addition of a simple \textsf{OnClient} annotation to a task. When this
annotation is present, the \iTask runtime automatically takes care of all
communication between the client and server parts of the task. The client part
of the task is executed by the \Sapl interpreter, which is available as a \Java 
applet on the client.

\subsection{Why switch to \JS?}
Until now we used the \Java Applet implementation of the \Sapl interpreter to
run \Clean programs on the client side of web applications. However, this
approach has several drawbacks. First of all, the use of \Java Applets requires
that the \Java Virtual Machine (JVM) be installed, which is often infeasible in
environments where the user has no control over the configuration of his/her
system. Secondly, the JVM exhibits significant latency during start-up. And
third, the JVM might not even be available on certain platforms (on mobile
devices in particular).

In contrast to the JVM, a \JavaScript virtual machine is shipped with all major 
browsers for most major platforms (including mobile devices). Over the past few
years client-side processing has become more and more important, and as a result
much effort is spent on improving the speed of these \JS interpreters. One
significant improvement that has recently been made is the addition of
just-in-time \textsf{JIT} compilation techniques to modern \JS interpreters such
as Google V8 (which is used in the Google Chrome browser). The resulting gains
in speed have made \JS an interesting alternative for the implementation of the
\Sapl interpreter.

\subsection{Co-operation between the Server and Client}
A special feature of the \Sapl interpreter is that we can use a dedicated form
of \textsf{Clean} dynamics \cite{DYNAMICS} for it. This allows \Clean (e.g.
partial function applications) to be serialized to strings. These strings can
can subsequently be stored somewhere, and at a later moment be retrieved,
deserialized and executed.

We extended the dynamics features of \Clean in  such a way that it is also
possible to serialize an expression in a \Clean executable, deserialize it
in the \Sapl interpreter (running the corresponding \Sapl program), and execute
the expression there. This is allows the execution of a \Clean program to be
migrated from the server to the client, and thus to decide at run-time whether
to execute a task on the server or the client. We used this feature to
implement client-side event handling for interactive web applications.
\cite{iEditors}. 

\section{The \Sapl Programming Language and Interpreter}
\label{sapljs:sec:sapl}
\Sapl stands for \textbf{S}imple \textbf{A}pplication \textbf{P}rogramming
\textbf{L}anguage. The basic version of \Sapl provides no special constructs for
algebraic data types. Instead, they are encoded as ordinary functions.  Details 
on this encoding and its consequences can be found in \cite{JKP}.
Here we restrict ourselves to giving a few examples to show how it is realized.

We start with the encoding of the list data type, together with the \texttt{sum} function:
\begin{CleanCode}
Nil       f1  f2 = f1
Cons x xs f1  f2 = f2 x xs
sum          xxs = select xxs 0 (\x xs = x + sum xs)
\end{CleanCode}

The \texttt{select} keyword acts as a hint to the compiler that \texttt{xxs} is 
a constructor, which allows the generation of more efficient code (again see
\cite{JKP} for the details). Semantically, \texttt{select} acts as the identity
function. The remaining arguments are functions that act on the arguments of a
constructor (analog to clauses in a case-statement).

As a more complex example, consider a \Haskell function such as
\texttt{mappair}, which is based on the use of pattern matching:
  
\begin{CleanCode}
mappair f Nil          zs           = Nil 
mappair f (Cons x xs)  Nil          = Nil 
mappair f (Cons x xs)  (Cons y ys)  = Cons (f x y) (mappair f xs ys) 
\end{CleanCode}
This definition is transformed to the following \Sapl function (using the
above definitions for \texttt{Nil} and \texttt{Cons}).
\begin{CleanCode}
mappair f as zs 
= select as Nil (\x xs = select zs Nil (\y ys = Cons (f x y) (mappair f xs ys)))
\end{CleanCode}
%
\Sapl is used as an intermediate formalism for the interpretation of non-strict
purely functional programming languages such as \Haskell and \Clean. For \Clean,
we added a \Sapl back-end to the \Clean compiler that generates \Sapl code which
the interpreter can run. Recently, the \Clean compiler has been extended to
handle the compilation of \Haskell programs as well \cite{HASCLEAN}.

The \Sapl interpreter has been implemented upon both \textsf{C} and \Java. The
\Java version can be loaded into browsers that have the \Java Virtual Machine
(JVM) installed, and can be used for execution of \iTask tasks on the client
side.

\subsection{Some remarks on the definition of \Sapl}
\Sapl is very similar to the core languages of \Haskell and \Clean. 
Therefore, we chose not give a full definition of its syntax and semantics.
Rather, we only say something about its main characteristics and give a few
examples to illustrate these.

The only keywords in \Sapl are \texttt{let}, \texttt{if} and \texttt{select}.
Only constant \texttt{let} expressions are allowed (that may be cyclic, \Sapl
has no separate \texttt{letrec}). These may occur at the top level in a function 
and at the top level in arguments of an \texttt{if} and \texttt{select}.
\texttt{where} clauses are not allowed. $\lambda$-expressions may only occur as 
arguments to a \texttt{select}. All other $\lambda$-expressions should be
lifted to the top level. 

For readability we adopted a \Clean like type definition style in \Sapl. This
also allows for the generation of more efficient code (as will become apparent
in section \ref{sapljs:sec:sapljs}). 
Below, a few examples to illustrate these concepts:

\begin{CleanCode}
::Boolean = True | False
::List    = Nil  | Cons x xs
 
ones        = let os = Cons 1 os in os 
f a b       = let c = add a 1, d = add b 2 in add c d
fac n       = if (eq n 0) 1 (mult n (fac (sub n 1))) 
filter f xs = select xs Nil (\y ys = if (f y) (Cons y (filter f ys)) (filter f ys))
\end{CleanCode}

%Note that with the above definition of boolean we can use \texttt{select} instead of \texttt{if} in \Sapl. 

%We will not give a full definition of \Sapl, but only stipulate it in characteristics.
%he body of 
%\Sapl is described by the following abstract syntax definition:
%\begin{haskell}
%function    &::= identifier \{identifier\}* '\hspace{-1.5mm}=' expr\\
%expr        &::= application | \ '\lambda' \{identifier\}+ '\hspace{-1.5mm}=\hspace{-0.6mm}' expr\\
%application &::= factor \{factor\}*\\
%factor      &::= identifier | integer | \ '(' expr ')'
%\end{haskell}
%The following EBNF describes the syntax of \Sapl.
%\\ VOLLEDIGE EBNF OF ALLEEN EEN PAAR VOORBEELDEN




\section{A \JS based implementation for  \Sapl}\label{sapljs:sec:sapljs}
In Section \ref{sapljs:sec:motivation} we already motivated our choice for \JS as an implementation platform for \Sapl.
Of course, it is possible to build a \Sapl interpreter in \JS in an equivalent way to that of the \Java version.
But \JS offers a number of features that make it possible to do it in another and better way.
First, it is possible to add \JS code to a running \JS program on-the-fly by using the \textsf{eval} functions.
Second, it is possible to build function calls to \JS function dynamically by supply a function name and an array of arguments.
Third, \JS has no static type check. Normally we consider this as a serous drawback, but here we can use this to build heterogeneous arrays,
containing elements of different types.

Therefore, we have chosen to build a compiler that translates \Sapl code to \JS and that exploits the above mentioned possibilities.
In our translation scheme we map a \Sapl function onto a \JS function with the same name and number of arguments.
The most important issue in the translation is the representation of closures and constructors.
We decided to represent a closure: $f a_0 .. a_n$ by a \JS array: $[f,[a_0, .. ,a_n]]$.
A constructor:  $C a_0 .. a_n$ by a \JS array: $[k,[C,a_0, .. ,a_n]]$, where $k$ is a number that is equal to the 
to place in the original type definition of the constructor (see also below).
The translation scheme is as follows:

\begin{haskell}
T\llbracket f a_0 .. a_n = body\rrbracket  &\equiv &\ function f(a_0, .. ,a_n) = T\llbracket body \rrbracket \\ 
T\llbracket f a_0 .. a_n \rrbracket &\equiv  &\ [f, [T \llbracket a_0\rrbracket , .. ,T \llbracket a_n ] ]\\
T\llbracket ::t = C_0  args_0  |   &\equiv  &\    function C_0(args_0){return [0,[C_0,args_0]];} .. \\
\hspace{1.15cm}.. | C_0  args_0\rrbracket   & &\ function C_n(args_n){return [n,[C_n,args_n]];} \\ 
T\llbracket select t a_0 .. a_n\rrbracket &\equiv  &\ var rr = eval(t); switch(eval(t)) {case 0: }
\end{haskell}

\begin{align*}
T \llbracket k \rrbracket &\equiv k
\end{align*}

\begin{align*}
T \llbracket v \rrbracket &\equiv v
\end{align*}

\begin{align*}
T \left \llbracket \begin{array}{cccccc}
let(rec) & v_1    & = & B_1    &    & \\
         & \ldots & = & \ldots &    & \\
         & v_n    & = & B_n    & in & E
\end{array} \right \rrbracket &\equiv \begin{array}{l}
function\ (v_1, \ldots, v_n)\ \{ \\
\ \ \ \ var\ x_1 = \llbracket B_1 \rrbracket; \\
\ \ \ \ \ldots \\
\ \ \ \ var\ x_n = \llbracket B_n \rrbracket; \\
\\
\ \ \ \ return\ \llbracket E \rrbracket; \\
\}\ ()
\end{array}
\end{align*}

\begin{align*}
T\llbracket \lambda v_1 \: \ldots \: v_n \rightarrow E \rrbracket &= \begin{array}{l}
function\ (v_1, \ldots, v_n)\ \{ \\
\ \ \ \ return\ \llbracket E \rrbracket; \\
\}
\end{array}
\end{align*}

\begin{align*}
T \llbracket E_1 \: E_2 \: \ldots \: E_n \rrbracket &
    \equiv [T \llbracket E_1 \rrbracket, T \llbracket E_2 \rrbracket, \ldots, T \llbracket E_n \rrbracket]]
\end{align*}

\begin{align*}
T \llbracket op_1 \: E_1 \rrbracket &
        \equiv op_1 \: eval(T \llbracket E_1 \rrbracket) \\
T \llbracket E_1 \: op_2 \: E_2 \rrbracket &
        \equiv eval(T \llbracket E_1 \rrbracket \: op_2 \: eval(T \llbracket E_2 \rrbracket) \\
T \llbracket E_1 \: ? \: E_2 : E_3 \rrbracket &
        \equiv eval(T \llbracket E_1 \rrbracket) \: ? \: eval(T \llbracket E_2 \rrbracket) : eval(T \llbracket E_3 \rrbracket)
\end{align*}
%
Example translations:
\begin{CleanCode}
:: list = nil | cons x xs	
filter f xs = select xs nil (\a as = if (f a) (cons a (filter f as)) (filter f as))
\end{CleanCode}
%
Is translated to:
\begin{CleanCode}
function nil() {return [0, ['nil']];}
function cons(x, xs) {return [1, ['cons', x, xs]];}

function filter(f, xs) {
    var ys = Sapl.eval(xs);
    switch (ys[0]) {
    case 0:
        return nil();
    case 1:
        var a = ys[1][1],
            as = ys[1][2];
        return (Sapl.eval([f, [a]])) ? (cons(a, [filter, [f, as]])) : 
                                       (Sapl.eval([filter, [f, as]]));
    }
}\end{CleanCode}
The example shows how the structure of the function is preserved in the translation.
The \emph{eval} function is used to evaluate closures.\\
- it turns closures into function calls\\
- it writes the result at the first position of the closure to enable sharing\\
- primitive values are return ned unchanged\\
- a boxed value is unboxed\\
- built in functions on numbers, boolean, string, etc are executed by evaluating their arguments and applying the operation to them

\subsection{Normal Forms}
Above we described a straightforward compilation scheme for \Sapl to \JS, where function calls (closures) are transformed to arrays.
The only optimization we already made is replacing arithmetic expressions directly by there \JS counterparts 
and thereby forcing the evaluation of there arguments. 
In this way we may impose unwanted eager sub-expressions into a program. 
If a programmer does not want this he or she should use an indirection via an own-made auxiliary function.

The \textsf{eval} functions is used to turn a closure into a real \JS function call and reduces a closure to head-normal-form.
The \textsf{eval} functions has to do a case analysis on the structure of the closure expression.
A closure expression can either be a primitive value (integer, boolean, string), a boxed value, a constructor value or a real function application closure.
For the last case we measured that a direct \JS call is about 15 times faster than making the same call using the 
\textsf{Sapl.eval} function on the closure representing the call. This overhead is significant. 
Fortunately, in many cases we can do an analysis at compile time and replace a call of \textsf{eval} for a closure by the corresponding hard coded \JS.
This transformation replaces:
\begin{verbatim}
Sapl.eval([f,[a1,..,an])
\end{verbatim}
by:
\begin{verbatim}
f(a1,..,an)
\end{verbatim}
This may only be done if \textsf{f} is a known function (thus not a variable) and the number of applied arguments matches the
number of arguments of \textsf{f}.
This can be done at every place where an explicit \textsf{eval} call is done:
\begin{itemize}
\item The first argument of a \textsf{select} or \textsf{if}.
\item The arguments of an arithmetic operation.
\end{itemize}
But also at every place where a result is returned, because \textsf{eval} is called for this result immediately.

<<start of Eddy's part>>\\
A major source of overhead in the current implementation are the calls to the
evaluator function {\texttt eval}, generated by applying the translation scheme
$T$ to built-in operators. To illustrate this, consider the following definition
of the fibonacci function, $fib$, in SAPL:

\begin{CleanCode}
fib = \ n -> n < 2 ? 1 : fib (n - 1) + fib (n - 2)
\end{CleanCode}

Applying $T$ to this expression yields the following code in JavaScript:

\begin{CleanCode}
var fib = function (n) {
    return n < 2 ? 1 : eval([fib, [eval(n) - 1]]) + eval([fib, [eval(n) - 2]]);
}
\end{CleanCode}

Using this definition, the number of calls to {\texttt eval} is proportional to
$O(n^2)$, which is prohibitive even for small values of $n$.


In the above example, two thunks are created for the recursive applications of
$fib$, only to be immediately reduced again to normal form because their value
is required by the operator $+$. It should be obvious that these applications
might as well be performed immediately in this case, thus avoiding both the
construction of the thunks and the outer two calls to {\texttt eval}. In
general, $T$ generates a call to {\texttt eval} when an expression is required
to be in normal form (for instance because it is used as an operand to a
built-in operator). However, if it is known at compile time that an expression
already is in normal form, the call to {\texttt eval} is unnecessary, and can
be avoided.

To take advantage of this observation, a new translation scheme, $S$, is
introduced, which is equivalent to $T$, except for the added restriction that
expressions generated by $S$ should always be in normal form. To adhere to this
restriction, the rule for function applications has to be rewritten as follows:

\begin{equation*}
S \llbracket E_1 \: E_2 \: \ldots \: E_n \rrbracket \equiv
\begin{cases}
[T \llbracket E_1 \rrbracket, [T \llbracket E_2 \rrbracket,
                               \ldots,
                               S \llbracket E_n \rrbracket]] &
        \text{if $n \le m$} \\
T \llbracket E_1 \rrbracket(T \llbracket E_2 \rrbracket,
                            \ldots,
                            T \llbracket E_n \rrbracket) &
        \text{if $n = m$} \\
eval([ T \llbracket E_1 \rrbracket(T \llbracket E_2 \rrbracket,
                                   \ldots,
                                   T \llbracket E_n \rrbracket),
                        [T \llbracket E_{m+1} \rrbracket,
                         \ldots,
                         T \llbracket E_n \rrbracket]]) &
        \text{otherwise (if $n > m$)}
\end{cases}
\end{equation*}
where $m$ is the arity of $E_1$.

The above rule assumes that $E_1$ is a lambda abstraction, and that the arity of
$E_1$ is known. The possibility that $E_1$ is actually an identifier introduces
the need for a symbol table, which allows names to be resolved to their
corresponding bindings. Furthermore, it is impossible to determine the arity of
a function at compile time in the following to the cases:

\begin{enumerate}
\item The function was passed as an argument to the current function
\item The function was returned as a the result of a function application
\end{enumerate}

In both cases, a call to {\texttt eval} is still necessary. The former case
occurs when $n > m$, and is already handled in the above rule. The latter case
occurs when $E_1$ is resolved to be a local identifier, and is handled by the
following additional rule:

\[ S \llbracket E_1 \: E_2 \: \ldots \: E_n \rrbracket \equiv
   eval(T \llbracket E_1 \: E_2 \: \ldots \: E_n \rrbracket) \]

Note how the behavior of $S$ closely mimics that of {\texttt eval}. Indeed,
from a conceptual point of view, $S$ performs the call to {\texttt eval} at
compile time, rather than run time. This allows the rule in $T$ for translating
built-in operators to be rewritten as follows:

\begin{align*}
T \llbracket op_1 \: E_1 \rrbracket &
        \equiv op_1 \: S \llbracket E_1 \rrbracket \\
T \llbracket E_1 \: op_2 \: E_2 \rrbracket &
        \equiv S \llbracket E_1 \rrbracket \: op_2 \:
               S \llbracket E_2 \rrbracket \\
T \llbracket E_1 \: ? \: E_2 \: : \: \rrbracket &
        \equiv S \llbracket E_1 \rrbracket \: ? \:
               S \llbracket E_2 \rrbracket \: : \:
               S \llbracket E_3 \rrbracket 
\end{align*}

Observe that this new rule does not generate any calls to {\texttt eval}, and
that $S$ only generates calls to {\texttt eval} if the arity of a function
cannot be determined at compile time.

In the above discussion, it was silently assumed that the result of a function
application is always in head normal form. To meet this requirement, however,
the rules for translating let(rec)-expressions and functions have to be
rewritten such that the expression returned is translated using $S$, rather than
$T$.




\section{Benchmarks} \label{sapljs:sec:benchmarks}
In this section we present the results of several benchmark tests for the  \JavaScript implementation of \Sapl (which we will call \Sapljs) and a
comparison with the \Java Applet implementation of \Sapl. 
We ran the benchmarks on a MacBook 2.26 MHz Core 2 Duo machine running MacOS X10.6.4.
We used Google Chrome using the V8 \JavaScript engine to run the programs.
At this moment V8 offers by far the fastest platform for running \Sapljs programs.
However, there is a heavy competition on \JavaScript engines and they tend to become much faster.
The benchmark programs we used for the comparison are a subset of the benchmarks  we used for comparing 
\Sapl with other interpreters and compilers in \cite{JKP}. In that comparison it turned out that \Sapl is at least twice as fast (and often even faster)
as other interpreter like \textsf{Helium} , \textsf{Amanda}, \textsf{GHCi} and \textsf{Hugs}.
Here we used the \Java Applet version for the comparison. This version is about 40\% slower than the \C  version
of the interpreter described in \cite{JKP} (varying from 25 to 50\% between benchmarks), but is still faster than other interpreters like \textsf{Amanda}, \textsf{Helium} and \textsf{GHCi} and \textsf{HUGS}.
The \Java Applet and \JavaScript  version of  \Sapl  and all benchmark code can be found at \cite{SAPL}.
We briefly repeat the description of the benchmark programs here.

\begin{enumerate}
%\setlength{\itemsep}{0mm}
\item {\bf Prime Sieve} The prime number sieve program, calculating the 2000th
prime number.
\item {\bf Symbolic Primes} Symbolic prime number sieve using Peano numbers,
calculating the 160th prime number.
\item {\bf Interpreter} A small \Sapl  interpreter. As an example we coded the prime
number sieve for this interpreter and calculated the 30th prime number.
\item {\bf Fibonacci} The (naive) Fibonacci function, calculating {\em fib 35}.
\item {\bf Match} Nested pattern matching (5 levels deep) repeated 160000 times.
\item {\bf Hamming} The generation of the list of Hamming numbers (a cyclic
definition) and taking the 1000th Hamming number, repeated 1000 times.
\item {\bf Sorting} Tree Sort (3000 elements), Insertion Sort (3000 elements), Quick Sort (3000 elements), Quick Sort (3000 elements) Merge
Sort (10000 elements, merge sort is much faster, we therefore use a much larger example)
\item {\bf Queens} Number of placements of 11 Queens on a 11 * 11 chess board.
\item {\bf Knights} Finding a Knights tour on a 5 * 5 chess board.
\item {\bf Prolog} A small Prolog interpreter based on unification only (no
arithmetic operations), calculating ancestors in a four generation family tree,
repeated 100 times.
\item {\bf Parser Combinators} A parser for Prolog programs based on Parser
Combinators parsing a 3500 lines Prolog program.
\end{enumerate}
%
For sorting a list of size $n$ we used a source list consisting of numbers $1$
to $n$. The elements that are 0 modulo 10 are put before those that are 1 modulo
$10$, etc.

%We could no use all of the previous used b 
The benchmarks cover a wide range of aspects of functional programming (lists, laziness, 
deep recursion, higher order functions, cyclic definitions, pattern matching, 
heavy calculations, heavy memory usage).
All times are machine measured. The programs where chosen in such a way that
they ran for at least a second or longer, if possible. This to eliminate start-up 
effects and to give the JIT compiler enough time to do its work. 
The output was always converted to a single number (e.g. 
by summing the elements of a list) to eliminate the influence of slow output 
routines.

\subsection{Benchmark Tests}
We ran several tests for several version of \Sapljs to evaluate the effectiveness of several
optimizations. 
We distinguish the following versions:
\begin{itemize}
\item{\Sapl} The \Java Applet version of \Sapl.
\item{\Sapljs plain} The  version using the simple compilation scheme and not using the conversion of arithmetical expressions 
to their \JS counterparts, but instead using built-in functions the handle them. 
\item{\Sapljs arith} The version using the simple compilation scheme with the conversion of arithmetical expressions.
\item{\Sapljs opt} The version using the static detection of evals of closures.
\item{\Sapljs strict} The previous version, including and using programmer made strictness annotations.
\end{itemize}

As an example to illustrate the effect of the different compilation schemes, consider the following \Sapl program and its translations.

- sapl program

- plain compilation

- plain with arith

- optimized

- with strictness annotations

\begin{table}\begin{center}
\caption{Speed Comparison  (Time in seconds)}
\footnotesize
\begin{tabular} {|l|l|l|l|l| l|l|l| l|l|l|l|l|} \hline
           & Pri & Sym & Inter & Fib & Match & Ham  & Qns & Kns & Tsort  & Sort & Parse & Plog  \\ \hline
\Sapl  & 6.1 & 17.6 & 7.8  & 7.3 & 8.5  & 15.7 & 7.9  & 6.5 & 47.1 & 4.4 & 4.0   & 16.4   \\ \hline
\Sapljs plain  & 4.3 & 13.2 & 6.0  & 6.5 & 5.9  & 9.8  & 5.6  & 5.1 & 38.3 & 3.8 & 2.6   & 10.1   \\ \hline
\Sapljs arith        & 2.0 & 1.7 & 8.2   & 4.0 & 4,1  & 8.4  & 6.6  & 3.7 & 17.7 & 2.8 & 0.7   & 4.4     \\ \hline
\Sapljs opt    & 0.9 & 1.5 & 1.8   & 0.2 & 1.0  & 4.0  & 0.1  & 0.4 & 5.7  & 1.9 & 0.4   & 3.2     \\ \hline
\Sapljs strict  & 0.9 & 0.8 & 0.8   & 0.2 & 1.4  & 2.4  & 2.4  & 0.4 & 3.0  & 4.5 & 0.4   & 1.6     \\ \hline \% mem  & 0.9 & 0.8 & 0.8   & 0.2 & 1.4  & 2.4  & 2.4  & 0.4 & 3.0  & 4.5 & 0.4   & 1.6     \\ \hline
\end{tabular}
\normalsize
\label{sapljs:table}

\end{center}
\end{table}

%\input{memTests}

\subsection{Evaluation of the Benchmark tests}
Before analysing the result we first make some general remarks about the performance of \Java, \JS and the \Sapl interpreter which are relevant for
a better understanding of the results. In general it is difficult to give absolute figures when comparing the speed of language implementations. 
They often depend on the platform (processor), the used operating system running on it and the benchmarks used to compare. 
Therefore, all numbers given represent rather rough estimations and should be used as global indications. 
For example, the language shoot-out site (add reference) indicates that \Java 
programs run between 3 and 5 times faster than \JS programs running on V8. 
Our own measurements on the used platform indicate that in our case the factor is about 3.

For a lazy functional language the creation of closures and the re-collection of them later on, often take a substantial part of program run-times.
It is therefore important to do some special tests that say something the speed of memory (de-)allocation.
The \Sapl interpreter uses a dedicated memory management unit (see \cite{JKP}) not depending on \Java memory management. 
The better performance of the \Sapl interpreter in comparison with the other interpreters partly depends on its fast memory management.
For the \JS implementation we rely on  the memory management of \JS itself.
We did some dedicated tests that showed that memory allocation for the \Java\ \Sapl interpreter is about 4-6 times as fast as for the \JS implementation.

Say something about percentage of time spend on memory management for benchmarks.

We could not run all benchmarks as long as we wished because of stack limitation for V8 \JS in Google Chrome. 
It support a standard stack of only 30k at this moment.
This is certainly enough for most \JS programs, but not for a number of our benchmarks that can be deeply recursive. 
Also forced \textsf{eval} calls can become deeply recursive. 
This limited the size of the runs of the following benchmarks: interpreter and all sorting benchmarks. 
Another benchmark we used previously: \textsf{twice twice twice twice inc 0}, could not be run at all.

For many \Sapljs examples a substantial part of their run-time is spent on memory management. 
They can only run significantly faster after a more efficient memory management is realized.
It is tempting to implement a memory management similar to that of \Sapl. 
But this memory management relies heavily on representing graphs by binary trees, 
which does not fit with our model for turning closures into \JS function calls which depends heavily on using arrays to
represent graphs.

We may conclude that memory management take a considerably percentage of execution time for the \Sapljs programs than for
the \Sapl interpreter. This explains why a number of the benchmarks execute slower than for the \Sapl interpreter, despite the fact
that the resulting code can be compiled instead of being interpreted as in \Sapl.

For example, \textsf{fibonacci} natively implemented in \Java runs 3 times as fast than programmed natively for \JS.
Adding strictness annotations in the \Sapl code of \textsf{fibonacci} takes care that the \JS translation result in the same code 
as the native one. 

\subsection{Alternative Optimizations}
In \JS it is possible to explicitly make curried versions of functions of mare than 1 argument.


\section{Optimizations}\label{sapljs:sec:optimizations}
Above we described a straightforward compilation scheme for \Sapl to \JS, where function calls (closures) are transformed to arrays.
The only optimization we already made is replacing arithmetic expressions directly by there \JS counterparts 
and thereby forcing the evaluation of there arguments. 
In this way we may impose unwanted eager sub-expressions into a program. 
If a programmer does not want this he or she should use an indirection via an own-made auxiliary function.

The \textsf{eval} functions is used to turn a closure into a real \JS function call and reduces a closure to head-normal-form.
The \textsf{eval} functions has to do a case analysis on the structure of the closure expression.
A closure expression can either be a primitive value (integer, boolean, string), a boxed value, a constructor value or a real function application closure.
For the last case we measured that a direct \JS call is about 15 times faster than making the same call using the 
\textsf{Sapl.eval} function on the closure representing the call. This overhead is significant. 
Fortunately, in many cases we can do an analysis at compile time and replace a call of \textsf{eval} for a closure by the corresponding hard coded \JS.
This transformation replaces:
\begin{verbatim}
Sapl.eval([f,[a1,..,an])
\end{verbatim}
by:
\begin{verbatim}
f(a1,..,an)
\end{verbatim}
This may only be done if \textsf{f} is a known function (thus not a variable) and the number of applied arguments matches the
number of arguments of \textsf{f}.
This can be done at every place where an explicit \textsf{eval} call is done:
\begin{itemize}
\item The first argument of a \textsf{select} or \textsf{if}.
\item The arguments of an arithmetic operation.
\end{itemize}
But also at every place where a result is returned, because \textsf{eval} is called for this result immediately.


Examples

 


\section{Related Work}
In this paper we extended the \textsf{iTask} toolkit with a generic framework for the inclusion of plug-ins, with the possibility to make calls from the plug-in to  \textsf{Clean} functions that can be executed on either client or server. 
We are not aware of any other functional system that has these features. 
However, there are functional approaches for handling web pages using the same formalism for server and client-side processing. Most of them compile to \textsf{JavaScript} for  client-side execution. 
An example of this approach is \textsf{Hop} \cite{serm06:WebProgrammingWithHop,loif07:HopClientSideCompilation}. \textsf{Hop} is a dedicated web programming language and its syntax is \textsf{HTML}-like. In \textsf{Hop} it is also possible to specify a complete web application without the (direct) use of \textsf{JavaScript}. \textsf{Hop} uses two compilers, one for compiling the server-side program and one for compiling the client-side part. The client-side part is only used for executing the user interface. The application essentially runs on the client and may call services on the server. \textsf{Hop} uses syntactic constructions for indicating client and server part code. It is build on top of the Scheme programming language. In our case we do not have to extend \textsf{Clean}, but can write the entire web application in \textsf{Clean} itself. 
In \cite{loif07:HopClientSideCompilation} it is shown that a reasonably good performance for the client-side functions in \textsf{Hop} can be obtained. For us, compiling to \textsf{JavaScript} is no option because \textsf{Clean} is lazy. Instead we use the \Sapl interpreter, which also has competitive performance as was shown in \cite{janj06:EfficientInterpretationDataTypesPatternsToFunctions_TFP06_SelectedPapers}  (chapter \ref{chap:SAPL}) and the graphics editor application.

\textsf{Links} \cite{cooe06:WebProgrammingWithLinks} and its extension \textsf{formlets}  is a functional language-based web programming language. \textsf{Links} compiles to \textsf{JavaScript} for rendering \textsf{HTML} pages, and SQL to communicate with a back-end database. A \textsf{Links} program stores its session state at the client side. In a Links program, the keywords \texttt{client} and \texttt{server} force a top-level function to be executed at the client or server respectively. In \textsf{Links}, processes can be spawned and these processes can communicate via message passing. Client-server communication is implemented using \textsf{Ajax} technology, like we do. 
In the \textsf{iData} and \textsf{iTask} toolkits, forms are generated generically for every data type, whereas in \textsf{Links} and \textsf{Formlets} these need to be coded by the programmer. 

The \textsf{Flapjax} language \cite{FLAPJAX} is an implementation of functional reactive programming in \textsf{JavaScript}, with features comparable to those of \textsf{Hop}. Both are designed to create intricate web applications. In \textsf{Flapjax}, \textsf{Hop} and \textsf{Formlets} processing is directly attached to web form handling, which is comparable to the use of call-backs in \textsf{iEditors}. 

A much more restricted approach has been implemented in \textsf{Curry} \cite{hanm07:PPDP}: only a very restricted subset of \textsf{Curry} is translated to \textsf{JavaScript} to handle client-side verification code fragments only.

Summarizing the main differences with the other approaches are:
\begin{itemize}
	\item \textsf{iTask}/\textsf{iEditor} applications are just plain \textsf{Clean} applications, where web forms are generated from data types. The other approaches define dedicated web languages where processing is attached to web forms;
	\item We can use the full \textsf{Clean} functionality at the client side because the \Sapl interpreter offers a full Clean platform. The other approaches rely on compilation to \textsf{JavaScript} with, in many cases, restrictions on the functions that can be compiled to \textsf{JavaScript};
	\item \textsf {Clean-SAPL} dynamics offers a generic and flexible way to attach call-back handling to web forms and plug-ins. 
	%start new
	Where the other approaches use static annotations to indicate whether functions have to be executed on either client or server, in our approach this can be decided dynamically, depending on the events to be processed.
\end{itemize}


\section{Conclusion and Future Work}\label{sapljs:sec:conclusions}
In this paper we evaluated the use of \JS as a target language for lazy functional programming languages like \Haskell of \Clean using the intermediate language \Sapl.
The implementation realized has the following characteristics:

\begin{itemize}
\item It achieves a speed for compiled benchmarks 
that is competitive with that of the  \Sapl interpreter and other interpreters like \textsf{Amanda}, 
\textsf{Helium}, \textsf{Hugs} and \textsf{GHCi}.
This despite the fact that \JS has a more than 3 times slower execution speed than \Java, which was used for the \Sapl interpreter.
\item The implementation supports the full \Clean language (but not all libraries are supported).
We tested the implementation against a large number of \Clean programs compiled with \Clean to \Sapl compiler. 
\item The runtime speed of benchmarks is often dominated by memory operations. 
But in many cases this overhead could be significantly reduced by a simple optimization
that reduces the creation thunks.
\item It is expected that the use of strictness information will give a further increase of performance
for many benchmarks.
%\item The implementation is straightforward based on a simple transformation scheme and using a few simple to implement optimizations.
%\item Functions from the source are in one-one correspondence with \JS functions.
%\item The lazy semantics is maintained by the encoding of thunks by \JS arrays and using a special \textsf{eval} function to turn thunks in function calls.
%\item Primitive data types like integer, string, boolean and char in the compiled program are equivalent to their \JS counterparts, which simplifies interactions with native \JS functions.
\end{itemize}

\subsection{Future Work}
We have planned the following future work:
\begin{itemize}
\item The realization of  an even better performance by making use of the strictness information generated by the \Clean compiler.
\item Automatic conversion of other data types like Records, Arrays, etc, between \Sapl and \JS.
\item The addition of dedicated libraries to access the DOM, etc.
\item The creation of special client-side \iTask libraries.
\item Further improvements of performance by making better compile time analysis (e.g. \JS currying instead of building thunks),
the use of tail recursion, etc.
\end{itemize}

\section*{Acknowledgements}
The authors would like to thank L\'aszl\'o Domoszlai for helping us with converting a preliminary version of the interpreter to one supporting the full \Sapl language and
for realizing a nice web-based user interface for the interpreter.

\bibliographystyle{abbrv}
\bibliography{refs}
\end{document}



